\documentclass[a4paper,11pt]{article}

\usepackage[utf8]{inputenc}
\usepackage[T1]{fontenc}
\usepackage[francais]{babel}
\usepackage{amsmath,amssymb}
\usepackage{xspace}
\usepackage{graphicx}
\usepackage{verbatim}
\usepackage{listings}
\usepackage[usenames,dvipsnames]{color}
\usepackage{hyperref}

\title{Projet Unix}
\author{Nathan GUYOT \and Vincent LEFOULON}

% ===============
\begin{document}
% ===============
\maketitle

\section{Le code}

Dans les grandes lignes, le code s'organise autour d'une boucle : celle parcourant
les images du dossier source. Cette boucle se base sur les indices et non directement
sur les éléments (stockés dans un \verb+array+) afin de pouvoir plus facilement
récupérer les images adjacentes.

A chaque tour de boucle, on crée la miniature, puis génère la page d'aperçu et enfin ajoute l'image au fichier HTML.

\section{Le Makefile}

Le graphe des tâches est le suivant :

\includegraphics[width=320px]{MakefileGraph.png}

Chaque règle se contente d'appeler un script shell de la forme \verb+make-X.sh+, lequel analyse ses paramètres et exécute un des scripts implémentés dans la première version de la galerie.

\section{Mesure du temps d'exécution}

Avec six images dans le dossier source, la commande \verb+time make gallery+ affiche :

\begin{verbatim}
real  0m1.724s
user  0m1.010s
sys 0m0.157s
\end{verbatim}

La parallélisation donne :

\begin{verbatim}
$ make clean; time make -j 15 gallery
real  0m0.548s
user  0m1.257s
sys 0m0.137s

$ make clean; time make -j 12 gallery
real  0m0.521s
user  0m1.140s
sys 0m0.160s

$ make clean; time make -j 9 gallery
real  0m0.510s
user  0m1.183s
sys 0m0.127s

$ make clean; time make -j 6 gallery
real  0m0.526s
user  0m1.207s
sys 0m0.150s

$ make clean; time make -j 2 gallery
real  0m0.530s
user  0m0.937s
sys 0m0.077s
\end{verbatim}

On constate que la parallélisation diminue fortement le temps d'exécution. Les
performances commencent par augmenter avec le nombre de processus parallèles, puis
on atteint un seuil à partir duquel on a trop de processus par rapport aux opérations
réalisables en parallèle, ce qui coûte des ressources sans nous en faire économiser.

\section{Les données EXIF}

Les suffixes dans les Makefiles sont obsolètes, notamment parce qu'ils ne permettent pas ne
définir de dépendances. Dans le cas présent, les fichiers \verb+.c+ n'ont donc
pas pour dépendance le \verb+.h+ correspondant. On pourrait plutôt faire :

\begin{verbatim}
%.c: %.h

%.o: %.c
	$(CC) $(CFLAGS) -o $@ -c $<
\end{verbatim}

% =============
\end{document} 
% =============
